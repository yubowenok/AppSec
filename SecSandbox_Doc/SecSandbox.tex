% 
% sample.tex
%
%   This is LaTeX template to get you started
%   making problem-set solutions
%
 
\documentclass[11pt]{article}
\usepackage{amsmath}
\usepackage{algorithm2e}
\usepackage{graphicx}

\setlength{\oddsidemargin}{0in}
\setlength{\evensidemargin}{0in}
\setlength{\textheight}{9in}
\setlength{\textwidth}{6.5in}
\setlength{\topmargin}{-0.5in}

% Sample macros -- how you define new commands
% My own set of frequently-used macros has grown to many hundreds of lines
\newcommand{\Adv}{{\mathbf{Adv}}}       
\newcommand{\prp}{{\mathrm{prp}}}                  % How to define new commands 
\newcommand{\calK}{{\cal K}}
\newcommand{\outputs}{{\Rightarrow}}                
\newcommand{\getsr}{{\:\stackrel{{\scriptscriptstyle\hspace{0.2em}\$}}{\leftarrow}\:}}
\newcommand{\andthen}{{\::\;\;}}    % \, \: \; for thinspace, medspace, thickspace
\newcommand{\Rand}[1]{{\mathrm{Rand}[{#1}]}}       % A command with one argument
\newcommand{\Perm}[1]{{\mathrm{Perm}[{#1}]}}       
\newcommand{\Randd}[2]{{\mathrm{Rand}[{#1},{#2}]}} % and with two arguments

\newcommand{\sym}[1]{{\tt \bf {#1}}}
\newcommand{\abs}[1]{{$\left|{#1}\right|$}}

%%%%%%%%%%%%%%%%%%%%%%%%%%%%%%%%%%%%%%%%%%%%%%%%%%%%%%%%%%%%%%%%%%%%%%%%%%%
\title{\bf Building Secure Turing Complete Sandbox\\[2ex] 
       \rm\Large Assignment 1.1, CS9163 - Application Security}
\author{Bowen Yu, 0487810}
\date{\today}


\begin{document}
\maketitle
\large

%%%%%%%%%%%%%%%%%%%%%%%%%%%%%%%%%%%%%%%%%%%%%%%%%%%%%%%%
\section{Overview}
The sandbox reads a customized, simplified coding language that is similar to the assembly language so as to supports all programs that a Turing Machine can handle. The language specification and usage is described in details in Section~\ref{sec:usage}. The inputs of the sandbox include a program script file and a program data file. The program script is an interpreted language so that the script is parsed and executed statement by statement. The sandbox supports creating and destroying variables. The values of variables and constants can only be integers ranging from $-10^{9}$ to $10^{9}$. The sandbox has a limit of applicable memory in the system (65536KB or $2^{24}$ integers). Also for memory reasons, the sandbox only supports a maximum of 100,000 lines of code.

\section{Usage}
\label{sec:usage}
\subsection{Running the Sandbox}
To compile the sandbox, simply do {\tt make} under the sandbox directory in the terminal.\\
To run the sandbox, execute the executable file by {\tt ./sandbox arg1 arg2}, where {\tt arg1} is the name of the script file and {\tt arg2} is the name of the data file. Note that {\tt arg2} can be omitted when the program does not need a data file.\\
You may also run the sandbox using debug mode as {\tt ./sandbox -d arg1 arg2}. In debug mode, the sandbox will print out every executed line with its line number.

\subsection{Sandbox Variables}
The variables in the sandbox can be single variables or elements of an array. Single variables are referenced by variable names and array elements are referenced by array names plus brackets and element indices, e.g. {\tt myVar[5]}. If the index (numbered from 0) exceeds the array length, the program will stop and report error. You may also embed variables and arrays in the array brackets as the array index, e.g. {\tt a[b[2]]}, as long as the variables that are used for the index already exist. For this case, the sandbox implements a recursive interpretation of array expression and has a maximum recursive reference level of 1000. The total memory required for all variables and arrays shall not exceed the memory limit, otherwise the program will report error and exit.

\subsection{Sandbox Instructions}
The sandbox supports a set of instructions that provide common functionalities just as any other programming language. The table of the supported statements are listed below. The instructions are {\bf not} case sensitive. Note that each instruction shall be written on a single line in the program file and one instruction shall not exceed 256 characters (usually a line will not be that long except for a malicious script :p), otherwise program will fail. The detailed usage description of each instruction is given following the list.

\begin{center}
  \begin{tabular}{| l | p{10cm} | }
    \hline
Instruction & Description \\ \hline
ALLOC &  Create a variable or an array\\ \hline 
DEALLOC & Free a variable or an array\\ \hline
SET & Set the value of a variable or an array element\\ \hline
ADD & Add the value of a variable or an array element\\ \hline
SUB & Subtract the value of a variable or an array element\\ \hline
MUL & Multiply the value of a variable or an array element by another value\\ \hline
DIV & Divide the value of a variable or an array element by another value\\ \hline
INC & Increase a value by one\\ \hline
DEC & Decrease a value by one\\ \hline
JPT  & Set jump point\\ \hline
JMP & Jump to a jump point in the program\\ \hline
JE/JNE  & Jump under an equal/unequal condition\\ \hline
JG/JGE   & Jump under a greater/greater-or-equal condition\\ \hline
JL/JLE    & Jump under a less/less-or-equal condition\\ \hline
PRT & Print a variable or an array element on the screen\\ \hline
RD & Read a value from the program data into a variable or an array element\\ \hline
EXIT & Terminate the program immediately\\ \hline
  \end{tabular}
\end{center}

\subsubsection{ALLOC}
{\bf Usage:} {\tt ALLOC var $|$ ALLOC var[length]}\\
The variable name {\tt var} must only include letters (a-z, A-Z), numbers (0-9) or underscore `\_' and cannot exceed 32 characters. Variable names must begin with letter. The variable names are case sensitive. Realloc an existing variable will cause error. If there is insufficient memory, the program will stop and report error. It is allowed that the {\tt length} can be specified by variable or array element.
\subsubsection{DEALLOC}
{\bf Usage:} {\tt DEALLOC var}\\
The variable name must match one of an existing variable. Dealloc an non-existing variable will cause error.
\subsubsection{SET}
{\bf Usage:} {\tt SET var1 var2}\\
If referencing variables, you must make sure the variables exist. {\tt var1} and {\tt var2} are not necessarily variables but may be integer constants. However, letting {\tt var1} be a constant is meaningless.
\subsubsection{ADD, SUB, MUL, DIV}
{\bf Usage:} {\tt ADD/SUB/MUL/DIV var1 var2}\\
Add/subtract/multiply/divide value of {\tt var1} by {\tt var2}. {\tt var2} can be constant integers. The result is stored in {\tt var1}. Note that if the result of the arithmetic exceeds the integer range of the system, the program will stop and report error. Dividing a variable by zero is not allowed.
\subsubsection{INC, DEC}
{\bf Usage:} {\tt INC/DEC var}\\
Increase or decrease the variable by one. Note that if the result of the arithmetic exceeds the integer range of the system, the program will stop and report error.
\subsubsection{JPT}
{\bf Usage:} {\tt JPT jname}\\
Set a jump point named {\tt jname}. The jump point name {\tt jname} must only include letters (a-z, A-Z), numbers (0-9) or underscore `\_' and cannot exceed 32 characters. Jump point names must begin with letter. Set a jump point with an existing jump point name will cause error.
\subsubsection{JMP}
{\bf Usage:} {\tt JMP jpoint}\\
Jump to the jump point specified by jump point name {\tt jpoint} and continue execution. The {\tt jpoint} can also be an integer, meaning that jump to the next {\tt jpoint}th line (previous when integer {\tt jpoint} is negative). Note that instructions are executed line by line, therefore jumping to a jump point that defined in the future or has not been parsed yet is not allowed.
\subsubsection{JE/JNE/JG/JGE/JL/JLE}
{\bf Usage:} {\tt JE/JNE/JG/JGE/JL/JLE jpoint var1 var2}\\
Check if the variables referenced by {\tt var1} and {\tt var2} satisfy the jump condition {\tt var1 condition var2}. If true, jump to the jump point specified by jump point name {\tt jpoint}. Also {\tt jpoint} may be an integer. Note that instructions are executed line by line, therefore jumping to a jump point that defined in the future or has not been parsed yet is not allowed.
\subsubsection{PRT}
{\bf Usage:} {\tt PRT var}\\
Print the value of a variable or an array element onto the screen in a single line.
\subsubsection{RD}
{\bf Usage:} {\tt RD var}\\
Read an integer from the data file into the variable specified by {\tt var}.
\subsubsection{EXIT}
{\bf Usage:} {\tt EXIT}\\
Terminate the program immediately. Usually this is performed after encountering an error.

\subsection{Script Comments}
To add comments in a script, simply add double slashes {\tt //} before the comment text. Comments may follow an instruction with {\bf at least one space in between} or be on a separate line. 


\subsection{Example Programs}
Included in the packages are three example programs: Counter, Fibonacci and TuringMaching. The program details are described in the comments of the scripts.\\\\
The Counter program counts down from 10 to 1 and outputs them on the screen. (file {\tt counter\_prog})\\\\
The Fibonacci program computes the first ten Fibonacci numbers and outputs them on the screen. (file {\tt fibonacci\_prog})\\\\
{\bf Extra Credit:} The TuringMachine program is a classic Turing machine that deals with a symbol collection, a tape (looks infinite but finite due to the sandbox memory limit) with a deterministic finite automaton. Details of the TuringMachine program is given in a separate document {\tt TM\_document}. (file {\tt tm\_prog} and {\tt tm\_data})
\end{document}
